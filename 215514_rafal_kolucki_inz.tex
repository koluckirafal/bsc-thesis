\documentclass[a4paper,12pt]{article}

\usepackage{amsmath,amssymb}
\usepackage{mathtools}
\usepackage[MeX]{polski}
\usepackage[utf8]{inputenc}
\usepackage{tocloft}
\usepackage{lscape}
\usepackage{breqn}
\usepackage{asymptote}
\usepackage{multicol}
\usepackage[margin=2.5cm]{geometry}
\usepackage{float}
\usepackage{multicol}
\usepackage{indentfirst}
\usepackage{cite}
\usepackage{url}
\usepackage{todonotes}
\usepackage{dmcsTitle}
\usepackage{minted}
\usepackage{caption}
\linespread{1.3}
\setminted{fontsize=\footnotesize,breaklines}
\newenvironment{longlisting}{\captionsetup{type=listing}}{}

\author{Kołucki Rafał}
\indexnumber{215514}
\thesis{inżynierska}
\supervisor[mgr Zbigniew Kulesza]{dr inż. Piotr Zając}
\doubletitle{Optymalizacja dostępu procesora do~magistrali systemowej na~platformie RISC-V}{Optimalization of~processor's access to~the~system bus on~the~RISC-V platform}

\begin{document}

\maketitle

\renewcommand{\cftsecleader}{\cftdotfill{\cftdotsep}}
\tableofcontents

\vfill

\pagebreak

\section*{Streszczenie}
TODO: max 500 słów, po polski i po angielsku; ogólny zakres pracy, przedstawić główne uzyskane wyniki, zastosowane metody, nowatorstwo pracy i wyciągnięte wnioski.

\pagebreak

\section{Cele i założenia pracy}

Celem niniejszej pracy jest rozbudowa implementacji wybranej magistrali systemowej o wsparcie dla potokowego transferu danych oraz analiza wpływu tej zmiany na wydajność całego systemu. Jest to realizowane w celu przyspieszenia operacji transferu danych między kontrolerem magistrali a peryferiami.

W pierwszej kolejności opisana zostanie architektura przykładowego układu typu System on Chip, z naciskiem na rdzeń procesora przeznaczenia ogólnego oraz na magistralę systemową.

Rozdział drugi będzie opisywać wybrane narzędzia wykorzystane do zrealizowania prostego układu SoC. Układ ten będzie wykorzystywany w formie symulatora uruchamianego na komputerze klasy PC, jak i logiki wykonywanej na układzie programowalnych macierzy bramek FPGA. Oprócz tego wymienione zostaną biblioteki wykorzystane przy realizacji testów jednostkowych używanych do weryfikacji poprawności działania wybranych peryferiów.

W rozdziale trzecim będzie przedstawiony opis rozbudowy implementacji magistrali systemowej Wishbone B4, zaczynając od doboru optymalizacji z dostępnej specyfikacji, kończąc zaś na praktycznych przykładach w rozbudowywanych peryferiach komunikujących się poprzez magistralę.

Przedostatni rozdział pracy zawierać będzie wyniki testów jednostkowych oraz wydajnościowych zrealizowanych z użyciem symulatora i układu FPGA, natomiast w ostatnim rozdziale opisane będą wnioski wyciągnięte z realizacji oraz testowania optymalizacji.

\section{Systemy jednoukładowe}

\subsection{Wprowadzenie do komputerów jednopłytkowych}

Komputery jednopłytkowe, których istnienie zawdzięczyć można postępowi miniaturyzacji układów scalonych, to systemy komputerowe, które zawierają całą wymaganą funkcjonalność na pojedynczej płytce drukowanej. Od typowego systemu komputerowego różni je to, że zwykle nie posiadają gniazd rozszerzeń, które byłyby wykorzystywane do rozbudowania komputera o dodatkowe lub czasami nawet wymagane rozszerzenia, które byłyby potrzebne w wybranym zastosowaniu danego systemu komputerowego. Pozwala to na zredukowanie ceny kosztem ograniczonych możliwości rozbudowy komputera.

Pierwsze komputery jednopłytkowe powstały w latach 70-tych\cite{Ariza_2021}, często w formie zestawu do samodzielnego złożenia i uruchomienia. Były one tańszą alternatywą dla większych systemów komputerowych, oferując podstawowe peryferia wymagane do tworzenia oprogramowania i interakcji z nim.
W późniejszych czasach komputery jednopłytkowe były coraz częściej wykorzystywane w systemach przemysłowych, głównie z powodu większej integracji komponentów i większej wytrzymałości niż systemy złożone z większej ilości elementów takich jak karty rozszerzeń.

Po upowszechnieniu się standardów takich jak IBM PC\cite{ibmpc1981}, część komputerów jednopłytkowych powstających w późniejszym czasie zaczęły wykorzystywać komponenty takie jak procesory architektury x86. Umożliwiło to wytwarzanie i uruchamianie programów oryginalnie przeznaczonych dla komputerów zgodnych z tymi standardami. Obecnie jednak, ze względu między innymi na mniejsze zużycie prądu, najczęściej wykorzystuje się układy bazujące na procesorach architektury ARM, znanej z urządzeń mobilnych.

Obecnie komputery jednopłytkowe większość funkcjonalności mają zawarte w pojedynczym układzie, zwanym System on Chip (system jednoukładowy). Poprzez redukcję ilości elementów na płytce drukowanej możliwym było zredukowanie rozmiarów komputerów jednopłytkowych, czasami zbliżając się rozmiarami do karty bankowej.

Przykładami popularnych obecnie komputerów jednopłytkowych, zbudowanych z użyciem układów SoC, są:
\begin{itemize}
	\item Raspberry Pi\cite{rpi1bplus} (Raspberry Pi Trading, Ltd.),
	\item BeagleBoard\cite{beagleboard} (Texas Instruments, BeagleBoard.org Foundation),
	\item UDOO Bolt\cite{udoobolt} (SECO S.P.A.)
\end{itemize}

\todo[inline]{ZDJĘCIE: popularny komputer jednopłytkowy}

\subsection{Definicja układu System on Chip w porównaniu z mikrokontrolerem}

Układy typu System on Chip (SoC) to układy, które na jednym podłożu półprzewodnikowym integrują wszystkie komponenty składające się na kompletny system mikroprocesorowy. Dzięki wysokiemu stopniu integracji układy te pozwoliły na znaczną miniaturyzację urządzeń elektronicznych bazujących na systemach mikroprocesorowych, takich jak telefony komórkowe czy komputery przenośne, oraz na ograniczenie kosztów przy większej skali produkcji.

Pojedynczy układ SoC składa się z przynajmniej jednej jednostki centralnej, pamięci RAM i ROM pozwalających na uruchomienie kodu inicjalizującego układ, peryferiów pozwalających na komunikowanie się ze światem zewnętrznym oraz jednej lub więcej magistral, które łączą wcześniej wspomniane peryferia.

Układy SoC odróżniają się od mikrokontrolerów tym, iż zawierają w sobie jak największą funkcjonalność możliwą do zmieszczenia w określonym rozmiarze fizycznym, uwzględniając takie ograniczenia jak dostępny proces technologiczny czy pobór prądu. Mikrokontrolery za to posiadają ograniczoną ilość peryferiów oraz mocy obliczeniowej w celu redukcji kosztów, co jest ważne w przypadku urządzeń wymagających specyficznej funkcjonalności w ograniczonym budżecie.

Typowy układ SoC posiada przynajmniej jedną jednostkę centralną o stosunkowo dużej mocy obliczeniowej, które pozwalają na uruchamianie pełnoprawnych systemów operacyjnych, takich jak Linux czy Microsoft Windows. Oprócz tego wymagają dodatkowo zewnętrznych pamięci RAM i ROM, gdyż oprogramowanie, które będzie uruchamiane na danym układzie, wymaga często dużych ilości pamięci, której nie da się zintegrować w samym układzie SoC.

\todo[inline]{RYSUNEK: budowa systemu jednoukładowego - jednostka centralna, magistrala, peryferia}

Wśród różnych rodzin układów SoC, stosowanych w komputerach jednoukładowych, wymienić można:
\begin{itemize}
	\item OMAP\cite{omap3530} firmy Texas Instruments,
	\item Vortex86\cite{vortex86ex2} firmy DM\&P,
	\item JH71x0\cite{jh7110} firmy Shanghai StarFive Technology
\end{itemize}

\subsection{Opis wybranej implementacji procesora z ISA RISC-V}

Łatwa dostępność dokumentacji związanej z architekturą RISC-V spowodowała, że istnieje duża ilość projektów implementujących procesory z nią zgodne. Procesory te są w większości zoptymalizowane do uruchomienia na układach FPGA (Field Programmable Gate Array), różniąc się w większości wewnętrzną architekturą oraz udostępnioną funkcjonalnością.

Przykładowe projekty procesorów gotowych do wykorzystania w projektach bazujących na układach FPGA to:
\begin{itemize}
	\item PicoRV32\cite{picorv32} autorstwa Claire Wolf,
	\item Ibex\cite{ibex} od grupy lowRISC C.I.C.,
	\item rodzina OpenXuantie\cite{openxuantie} firmy T-Head Semiconductor
\end{itemize}

Na potrzeby tej pracy został wybrany projekt VexRiscv\cite{vexriscv:2018:Online}, który jest rdzeniem implementującym architekturę RISC-V\cite{Waterman:EECS-2014-54} w wariancie 32-bitowym (RV32I), zoptymalizowanym pod kątem układów FPGA. Jego modularność z użyciem rozszerzeń pozwala na konfigurację parametrów na etapie budowania rdzenia w zależności od wymogów projektu, takich jak ilość etapów potoku wykonywania rozkazów, wspierane rozszerzenia architektury czy wielkość pamięci podręcznej poziomu pierwszego.

Wśród dostępnych opcji konfiguracyjnych są takie elementy jak:
\begin{itemize}
	\item opcjonalne rozkazy m.in. mnożenia, operacji na liczbach zmiennoprzecinkowych zdefiniowanych dla architektury RISC-V,
	\item regulowana długość potoku wykonawczego od 2 do 5 stopni,
	\item możliwość podłączenia procesora do dowolnej magistrali, z czego dostępne są adaptery dla magistral AXI4, Avalon oraz Wishbone,
	\item opcjonalna pamięć podręczna programu i danych, każda w dowolnej ilości,
	\item opcjonalna jednostka zarządzania pamięcią (MMU)
\end{itemize}

Wybór tego procesora na potrzeby pracy został podyktowany przede wszystkim wcześniejszym doświadczeniem autora niniejszej pracy w pracy z tym projektem. Na potrzeby tej pracy użyta zostanie podstawowa konfiguracja. Ze względu na możliwe ograniczenia związane z platformą testową (na przykład rozmiar układu FPGA), wyłączone zostało wsparcie dla wszystkich opcjonalnych grup rozkazów. W zamian dodana zostanie pamięć podręczna danych, która odpowiedzialna będzie za buforowanie odczytywanych z potencjalnie wolniejszej pamięci operacyjnej danych.

\subsection{Przegląd dostępnych magistrali systemowych}

Magistrale, jak zostało wspomniane w poprzednich podrozdziałach, służą do transferu danych między poszczególnymi komponentami systemu. Od ich parametrów zależy sprawność komunikacji między peryferiami - źle skonfigurowana magistrala może spowalniać działanie całego systemu, przykładowo poprzez ograniczenie przepływności danych.

W celu wykorzystania danej szyny, wszystkie komponenty muszą mieć zaimplementowany interfejs, który pozwala na podłączenie się do danej magistrali. Jeśli któreś komponenty implementują interfejs innego mostka niż pozostałe, to można wykorzystać mostki, które pozwalają na konwersję między danymi standardami magistral.

Istnieją różne standardy magistrali przeznaczone dla układów SoC, wśród których bardziej znanymi standardami są:
\begin{itemize}
	\item AMBA AXI\cite{amba-axi:2021:Online} od Arm Ltd.,
	\item Avalon\cite{avalon:2005:Online} autorstwa firmy Altera,
	\item Wishbone\cite{wishbone:2019:Online} od Silicore Corporation
\end{itemize}

Każda z tych specyfikacji ma swoje unikalne cechy: przykładowo AMBA AXI definiuje niezależne kanały do odczytu i zapisu danych, pozwalając na jednoczesne wykonywanie tych dwóch operacji. Wishbone zaś definiuje wyłącznie sposób przekazywania danych między dwiema modułami, pomijając między innymi kwestię topologii połączeń między elementami.

W ramach tej pracy wykorzystywana będzie magistrala Wishbone ze względu na w pełni dostępną publicznie, wolną od tantiemów specyfikację oraz jej neutralność (nie została stworzona przez dużego producenta mikroprocesorów czy układów FPGA).

\subsection{Wnioski}

Jako, iż celem tej pracy jest poprawa wydajności systemu jednoukładowego, na początek wymagane było zdefiniowanie, czym ów system jest z uwzględnieniem tła historycznego oraz jaka jest typowa konstrukcja takiego układu. Po określeniu najważniejszych elementów takiego systemu dobrane zostały główne elementy, które będą wykorzystywane w niniejszej pracy: procesor VexRiscv, obsługujący rozkazy architektury RISC-V, oraz magistrala Wishbone, która definiować będzie sposób komunikacji między poszczególnymi elementami układu SoC.

\section{Projektowanie, testowanie i uruchamianie układu SoC}
\subsection{Konstruowanie układu SoC z użyciem frameworka LiteX}
O ile układ SoC można zaprojektować w całości od podstaw, częstą praktyką jest wykorzystywanie gotowych komponentów, często napisanych przez zewnętrznych dostawców, w celu szybszego zaprojektowania układu spełniającego założenia projektowe.

Framework LiteX\cite{https://doi.org/10.48550/arxiv.2005.02506} dostarcza różnego rodzaju komponenty oraz narzędzia pozwalające na zbudowanie zarówno kompletnego układu SoC jak i bardziej specjalizowanych systemów w formie syntezowalnej logiki. W tym celu wykorzystuje bibliotekę Migen, która implementuje język opisu sprzętu w formie obiektów w języku Python. Dzięki temu możliwe jest opisanie logiki wykorzystując popularny język programowania wykorzystujący znajomą składnię, co zmniejsza barierę wejścia; oprócz tego można wykorzystać narzędzia z ekosystemu języka Python, dzięki czemu dostępne są między innymi większe możliwości parametryzacji logiki czy systemy do przeprowadzania testów automatycznych. Jednocześnie możliwe jest łączenie logiki napisanej w innych językach, co ułatwia wykorzystywanie komponentów nie będących częścią frameworka.

\todo[inline]{Dalszy opis frameworka LiteX, opisanie projektowania od SoCCore do pełnego SoC z peryferiami; tez troche o builderze}

\subsection{Symulowanie układu z użyciem LiteX i Verilator}
LiteX zawiera modularne środowisko symulacji, które pozwala na wygenerowanie symulatora na podstawie opisanej w języku Migen logiki oraz parametrów dotyczących środowiska zewnętrznego. Możliwe jest dzięki temu przetestowanie układu oraz przeznaczonego na nie oprogramowania, przykładowo poprzez wykorzystanie wirtualnego interfejsu Ethernet do komunikacji. Ułatwia to usuwanie błędów z projektu układu jak i oprogramowania, zanim jeszcze układ zostanie zsyntetyzowany do formy bitstreamu dla układu FPGA.

W celu uruchomienia naszego systemu w symulatorze, układ musi być skonfigurowany do zbudowania z użyciem narzędzi symulacyjnych zamiast narzędziami syntezy dedykowanych konkretnemu układowi FPGA. W tym celu LiteX integruje obsługę programu Verilator w celu konwersji układu do kodu komputerowego w formie źródłowej. Ów program linkowany jest z tak zwanym "wrapperem", który łączy naszą logikę z dodatkowymi modułami, które symulują wybrane peryferia z użyciem zasobów systemu operacyjnego, na którym symulacja będzie wykonywana. W przypadku tego projektu będzie to interfejs szeregowy UART dostępny w formie standardowego strumienia wejścia/wyjścia oraz niewykorzystany interfejs Ethernet w formie wirtualnego interfejsu sieciowego w systemie hosta. W ten sposób symulacja układu nie różni się znacząco od wykonywania programu komputerowego, co ułatwia interakcję z programem wykonywanym w wirtualnym środowisku.

\todo[inline]{modele wspierane przez LiteX Sim}

W praktyce zbudowanie symulatora układu polega na podmianie wybranych peryferiów na modele, które są wspierane przez Verilator, oraz na opisaniu wykorzystywanych modułów oraz sposobu dostępu do nich z poziomu systemu operacyjnego hosta.

\todo[inline]{sim_config podawany do builder.build()}

\subsection{Testowanie logiki z użyciem biblioteki cocotb}
Standardową praktyką przy testowaniu układów logicznych jest pisanie tak zwanych test benchy, czyli skryptów wykorzystywanych do testowania logiki poprzez sterowanie sygnałami wejściowymi oraz porównywanie stanu sygnałów wyjściowych z oczekiwanymi wzorcami. Owe skrypty są zwykle pisane w tym samym języku, co testowany układ logiczny.

Ze względu na wykorzystywanie języka Python w tej pracy do zaprojektowania układu SoC, dobrym pomysłem jest użycie go również w celu testowania logiki. W tym celu użyta została biblioteka cocotb, która pozwala na testowanie projektów zaimplementowanych w językach (System)Verilog i VHDL z użyciem skryptów napisanych w języku Python. Cocotb współpracuje z zarówno komercyjnymi jak i otwarto-źródłowymi symulatorami, dzięki czemu można tą bibliotekę wykorzystać w istniejących już środowiskach do projektowania układów logicznych.

\todo[inline]{pisanie testbenchy w pythonie, napomknięcie o cocotbext-wishbone}

\section{Realizacja oraz testowanie optymalizacji wybranej magistrali systemowej}
TODO: co my dokładnie będziemy robić z tym wishbone?

\subsection{Rodzaje cykli na szynie Wishbone}
TODO: opis classic i registered cycles, tak jak w specce i na przykładach

\subsection{Implementacja podzespołu z obsługą cykli transmisji seryjnej do stałego adresu}
TODO: peryferium fifo, opis jak działa + logika od burstów

\subsection{Rozbudowanie wybranego podzespołu (SRAM) o obsługę cykli transmisji seryjnej z inkrementacją adresu}
TODO: soc/interconnect/wishbone.WishboneSRAM - opis działania + logiki burstów
\section{Realizacja testów i analiza ich wyników}

Po zmodyfikowaniu podzespołu należy sprawdzić, czy oryginalna funkcjonalność nie została w jakiś sposób naruszona oraz czy dodana funkcjonalność działa zgodnie z założeniami. W tym celu podzespół należy poddać procesowi weryfikacji. Można jej dokonać na różne sposoby, zaczynając od testów jednostkowych i integracyjnych, kończąc zaś na weryfikacji formalnej.
Weryfikacja formalna jest dokładniejsza ze względu na wykorzystywanie formalnych metod matematycznych, zajmuje jednak znacznie więcej czasu ze względu na konieczność opisania w formie matematycznej oczekiwanego zachowania układu. Wobec tego zostanie wykorzystana prostsza metoda testów jednostkowych - polega ona na opisaniu sekwencji sygnałów wejściowych testowanego komponentu oraz instrukcji pozwalających na interpretację stanów wyjściowych poprzez porównywanie ich do oczekiwanych wzorców.

\subsection{Analiza i porównanie przebiegów aktywności magistrali}

Wspomniane testy jednostkowe zostaną wykorzystane w kontekście tej pracy do sprawdzenia, czy dane są poprawnie odczytywane i zapisywane w pamięci Block RAM z wykorzystaniem zarówno klasycznych jak i potokowych cykli operacji na magistrali Wishbone. W tym celu każdy test wygeneruje sekwencję słów oraz adresów docelowych, które będą wykorzystywane zarówno do wytworzenia sygnałów wejściowych oraz jako wzorzec, do którego będzie porównywany stan końcowy testowanej pamięci.
Ze względu na to, iż obserwujemy wyłącznie stan zewnętrzny, do pamięci Block RAM dostanie dodany drugi port, który będzie pozwalał na dokonywanie operacji na pamięci z pominięciem interfejsu Wishbone.

\subsubsection{Implementacja testów jednostkowych dla peryferiów}

Testy zostały zrealizowane z użyciem języka Python, wykorzystując bibliotekę Cocotb, która umożliwia tworzenie środowiska testowego z użyciem składni języka Python zamiast w Verilogu. Pozwala to na integrację z bibliotekami ekosystemu języka Python, ułatwiając takie sprawy jak generowanie testowego zbioru danych.
Oprócz tego wykorzystane zostały biblioteki cocotb-test oraz cocotbext-wishbone. Cocotb-test integruje Cocotb z systemem testów automatycznych pytest, dzięki czemu napisane testy można uruchamiać z różnymi ustawieniami, co w połączeniu z parametryzacją samych testów umożliwia pokrycie różnych sposobów komunikacji z testowanym podzespołem na magistrali. Cocotbext-wishbone zaś implementuje funkcje do sterowania i monitorowania sygnałów na magistrali Wishbone, co pozwala zaoszczędzić czas poprzez wywoływanie gotowych funkcji realizujących operacje na magistrali zamiast poprzez ręczne sterowanie szyną.

\todo[inline]{LISTING: struktura katalogu test/}

Testowane peryferia, czyli pamięć SRAM oraz transceiver FIFO, zostały zainstancjonowane w minimalnym module LiteX, który na zewnątrz udostępnia wspólną magistralę Wishbone, drugi port pamięci SRAM, wyjście nadawczej kolejki FIFO oraz wejście odbiorczej kolejki FIFO. W ten sposób mamy dostęp do wszystkich sygnałów, które posłużą do wysterowania magistrali oraz komunikacji z peryferiami inną drogą, umożliwiając obserwowanie wpływu implementacji interfejsu Wishbone na komunikację z peryferiami. Jako że biblioteka Cocotb oczekuje opisu układu w języku Verilog, moduł LiteX jest eksportowany do pliku Verilog przed rozpoczęciem wykonywania testów.

\todo[inline]{LISTING: harness}

\subsubsection{Procedura wykonywania testów}
Zbiór testów jest wykonywany z użyciem komendy pytest, która jest częścią biblioteki o tej samej nazwie. Według deficji testów z pliku test.py generowana jest lista testów z różnymi kombinacjami parametrów według podanych przedziałów. Dzięki temu z czterech rodzajów testów (cykl klasyczny oraz wybrany cykl potokowych dla peryferium FIFO oraz modułu SRAM) otrzymano zbiór xxx testów, które zostają wykonywane po kolei. Po zakończeniu testów program pytest zwraca ich wyniki, które można zapisać również w formacie HTML. Umożliwia to wyfiltrowanie przypadków, dla których testy nie kończą się zgodnie z oczekiwaniami.

\todo[inline]{LISTING: struktura przykładowego testu}

\subsubsection{Porównanie wyników testów}

\todo[inline]{RYSUNEK: Przebiegi dla każdego rodzaju peryferiów i każdego rodzaju cykli}

\subsection{Testy wydajnościowe systemu jednoukładowego na symulatorze oraz na układzie FPGA}

Po zweryfikowaniu poprawności działania peryferiów kolejnym etapem jest przeprowadzenie testów wydajnościowych. Polegać one będą na wykonaniu testowego układu SoC, który zawierać będzie peryferium w wariancie zarówno zmodyfikowanym jak i oryginalnym. Na tym układzie uruchamiany będzie program, który będzie mierzyć prędkość operacji odczytu i zapisu zbioru danych w kolejności sekwencyjnej i losowej. Pozwoli to na zbadanie wpływu obsługi transferów potokowych na szybkość wykonywania operacji.

\subsubsection{Wykorzystane urządzenia}

Testy wydajnościowe były wykonywane w dwóch środowiskach: na zestawie deweloperskim Digilent Arty A7-35, zawierającym układ programowalnych bramek logicznych Artix-7 firmy Xilinx, oraz na symulatorze uruchomionym na średniej klasy komputerze klasy PC będącym pod kontrolą systemu operacyjnego Debian GNU/Linux. Symulator został automatycznie zbudowany poprzez framework LiteX z użyciem programu Verilator, zaś plik konfigurujący zachowanie układu FPGA (zwany bitstreamem) zbudowany został z wykorzystaniem zbioru narzędzi do syntezy Vivado firmy Xilinx.

\todo[inline]{RYSUNEK: zdjęcie arty-a7, output z symulatora lub inne jego przedstawienie}

\subsubsection{Testowy układ SoC i jego architektura}

Testowy system mikroprocesorowy składa się z procesora VexRiscv, czterech bloków pamięci SRAM, interfejsu UART do komunikacji między programem testowym a użytkownikiem oraz niewykorzystanych bloków interfejsu Ethernet i analizatora stanów logicznych.
Procesor VexRiscv został skonfigurowany z pamięcią podręczną danych wielkości XX kibibajtów oraz z obsługą architektury RV32I - obecność pamięci podręcznej jest szczególnie ważna, gdyż to właśnie na jej potrzeby procesor zawiera wsparcie dla transferów potokowych na magistrali Wishbone.
Z czterech pamięci SRAM dwa są w trybie tylko do odczytu - zawierać będą program rozruchowy oraz program testowy, które załadowane będą jako część bitstreamu dla układu FPGA.

\todo[inline]{RYSUNEK: schemat blokowy TestSoC}

\subsubsection{Program dla jednostki centralnej RISC-V badający szybkość operacji na pamięci}

Program testowy, zaimplementowany w języku C, testuje wybrany obszar pamięci poprzez zapisanie oraz odczytanie całego obszaru. Test ten jest wykonywany na dwa sposoby: dokonując operacji na kolejnych oraz losowy adresach.
Funkcje odpowiadające za przeprowadzenie testu są dostępne w ramach biblioteki libbase, będącej częścią frameworka LiteX - o ile funkcje można wywołać z poziomu konsoli programu rozruchowego, to program na etapie budowania zapisuje adres startowy oraz pojemność testowanej pamięci przekazany przez skrypt budujący gateware, pozwalając na automatyczne wykonanie testu w momencie załadowania bitstreamu na układ FPGA lub uruchomienia symulatora.
Dodatkową funkcjonalnością, dostępną w przypadku uruchomienia testu w symulacji, jest śledzenie wykonywania poprzez możliwość zapisywania "flagi" pod określonym wcześniej adresem - zawartość tego adresu jest potem zapisywana w pliku z przebiegami sygnałów emulowanego systemu, dzięki czemu można zbadać, jakie operacje na magistrali są wykonywane w danym fragmencie testu.

\todo[inline]{struktura programu, sposób kompilacji?}

\subsubsection{Wyniki testów wydajnościowych}

\todo[inline]{tabelki i wstępne spostrzeżenia}

%\subsection{CFU Plauground - w razie jakby nie udało się wycisnąć mininalnej ilości tekstu}

\subsection{Wnioski po realizacji testów}

\section{Podsumowanie}

Na początku pracy omówione zostały systemy jednoukładowe, zaczynając od historii komputerów jednoukładowych oraz wymienienia popularnych obecnie urządzeń bazujących na układach System on Chip. Po określeniu w ten sposób przykładów takich systemów możliwe było określenie wspólnych cech, które definiują, czym dokładnie jest system jednoukładowy. Określona została również wybrana implementacja procesora zgodnego z architekturą RISC-V oraz standardu magistrali systemowej, na których została oparta platforma testowa stworzona na potrzeby niniejszej pracy.

Opisane zostało zoptymalizowanie wydajności systemu jednoukładowego poprzez zaimplementowanie obsługi transferów potokowych w module pamięci RAM. Optymalizacja ta pozwoliła na przyspieszenie odczytu danych z pamięci RAM o przynajmniej 23 procent bez modyfikacji innych parametrów pracy systemu, takich jak częstotliwość taktowania. Jednocześnie według testów jednostkowych słowa były przekazywane w niemal połowie mniejszej ilości cykli niż w wypadku pojedynczych transferów. Było to możliwe dzięki przeanalizowaniu możliwości wybranego procesora oraz zaznajomieniu się z różnymi trybami pracy opisanymi w specyfikacji magistrali systemowej; jednocześnie możliwym jest dalsze analizowanie możliwości optymalizacji systemów jednoukładowych w celu osiągnięcia lepszych parametrów wydajnościowych.

Zarówno moduł SRAM, będący częścią większego środowiska LiteX wykorzystywanego do tworzenia systemów jednoukładowych, jak i testy jednostkowe wykorzystane w celu sprawdzenia poprawności działania zmodyfikowanego układu, zostały zaimplementowane w tym samym języku programowania o nazwie Python. Umożliwiło to wykorzystanie tych samych narzędzi, takich jak systemy testów jednostkowych, których używa się do rozwoju oprogramowania w tym języku, zmniejszając barierę wejścia w porównaniu z narzędziami dostępnymi dla częściej wykorzystywanych języków opisu sprzętu, takich jak Verilog czy VHDL. Jednocześnie, dzięki integracji środowiska LiteX z zamkniętymi narzędziami dedykowanymi dla konkretnych układów FPGA, możliwym było uruchomienie platformy testowej na fizycznym układzie FPGA.

Praca ta udowadnia, iż jest możliwym realizacja kompletnego systemu jednoukładowego z wykorzystaniem w większości otwartoźródłowych komponentów, co umożliwia ograniczenie kosztów związanych z projektowaniem i integracją takiego systemu. Opisana optymalizacja została włączona do publicznego kodu środowiska LiteX, co pozwala na wykorzystanie jej w rozwiązaniach zarówno hobbystycznych, jak i komercyjnych.


\bibliographystyle{plain}
\bibliography{refs}

\end{document}
