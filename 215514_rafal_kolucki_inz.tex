\documentclass[a4paper,12pt]{article}

\usepackage{amsmath,amssymb}
\usepackage{mathtools}
\usepackage[MeX]{polski}
\usepackage[utf8]{inputenc}
\usepackage{tocloft}
\usepackage{lscape}
\usepackage{breqn}
\usepackage{asymptote}
\usepackage{multicol}
\usepackage[margin=2.5cm]{geometry}
\usepackage{float}
\usepackage{multicol}
\usepackage{indentfirst}
\usepackage{cite}
\usepackage{url}
\usepackage{todonotes}
\usepackage{dmcsTitle}
\usepackage{minted}
\usepackage{caption}
\linespread{1.3}
\setminted{fontsize=\footnotesize,breaklines}
\newenvironment{longlisting}{\captionsetup{type=listing}}{}

\author{Kołucki Rafał}
\indexnumber{215514}
\thesis{inżynierska}
\supervisor[mgr Zbigniew Kulesza]{dr inż. Piotr Zając}
\doubletitle{Optymalizacja dostępu procesora do~magistrali systemowej na~platformie RISC-V}{Optimalization of~processor's access to~the~system bus on~the~RISC-V platform}

\begin{document}

\maketitle

\renewcommand{\cftsecleader}{\cftdotfill{\cftdotsep}}
\tableofcontents

\vfill

\pagebreak

\section*{Streszczenie}
TODO: max 500 słów, po polski i po angielsku; ogólny zakres pracy, przedstawić główne uzyskane wyniki, zastosowane metody, nowatorstwo pracy i wyciągnięte wnioski.

\pagebreak

\section{Cele i założenia pracy}

Celem niniejszej pracy jest rozbudowa implementacji wybranej magistrali systemowej o wsparcie dla potokowego transferu danych oraz analiza wpływu tej zmiany na wydajność całego systemu. Jest to realizowane w celu przyspieszenia operacji transferu danych między kontrolerem magistrali a peryferiami.

W pierwszej kolejności opisana zostanie architektura przykładowego układu typu System on Chip, z naciskiem na rdzeń procesora przeznaczenia ogólnego oraz na magistralę systemową.

Rozdział drugi będzie opisywać wybrane narzędzia wykorzystane do zrealizowania prostego układu SoC. Układ ten będzie wykorzystywany w formie symulatora uruchamianego na komputerze klasy PC jak i logiki wykonywanej na układzie reprogramowalnym FPGA. Oprócz tego wymienione zostaną biblioteki wykorzystane przy realizacji testów jednostkowych używanych do weryfikacji poprawności działania wybranych peryferiów.

W rozdziale trzecim będzie przedstawiony opis rozbudowy implementacji magistrali systemowej Wishbone B4, zaczynając od doboru optymalizacji z dostępnej specyfikacji, kończąc zaś na praktycznych przykładach w rozbudowywanych peryferiach komunikujących się poprzez magistralę.

Przedostatni rozdział pracy zawierać będzie wyniki testów jednostkowych oraz wydajnościowych zrealizowanych z użyciem symulatora oraz układu FPGA, zaś w ostatnim rozdziale opisane będą wnioski wyciągnięte z realizacji oraz testowania optymalizacji.

\section{Budowa systemu jednoukładowego}
Układy typu System on Chip to układy, które na jednym podłożu półprzewodnikowym integrują wszystkie komponenty składające się na kompletny system mikroprocesorowy. Dzięki wysokiemu stopniu integracji układy te pozwoliły na znaczną miniaturyzację urządzeń elektronicznych bazujących na systemach mikroprocesorowych, takich jak telefony komórkowe czy komputery przenośne, oraz na ograniczenie kosztów przy większej skali produkcji.

Pojedynczy układ SoC składa się z przynajmniej jednej jednostki centralnej, pamięci RAM i ROM pozwalających na uruchomienie kodu inicjalizującego układ, peryferiów pozwalających na komunikowanie się ze światem zewnętrznym oraz magistrali, która łączy wszystkie wcześniej wspomniane peryferia.

Układy SoC odróżniają się od mikrokontrolerów tym, iż posiadają jednostki centralne o stosunkowo dużej mocy obliczeniowej, które pozwalają na uruchamianie pełnych systemów operacyjnych, takich jak Linux czy Microsoft Windows. Oprócz tego wymagają zewnętrznych pamięci RAM i ROM, gdyż nie integrują wymaganych ilości pamięci na potrzeby systemów operacyjnych i programów.

\todo[inline]{rysunek przedstawiający budowę typowego SoC - jednostka centralna, magistrale, peryferia}

\subsection{Opis wybranej implementacji procesora z ISA RISC-V}
Na potrzeby tej pracy został wybrany projekt VexRiscv\cite{vexriscv:2018:Online}, który jest rdzeniem implementującym architekturę RISC-V w wariancie 32-bitowym, zoptymalizowanym pod kątem układów FPGA. Jego modularność z użyciem rozszerzeń na etapie budowania rdzenia pozwala na konfigurowanie parametrów w zależności od wymogów projektu, takich jak ilość etapów potoku wykonywania rozkazów, wspierane rozszerzenia architektury czy wielkość pamięci cache poziomu pierwszego.

\todo[inline]{dokładniejszy opis procesora: ogólny schemat budowy, jakieś dodatkowe pomysły?}

\subsection{Przegląd dostępnych magistrali systemowych}

Magistrale, jak zostało wspomniane w poprzednich podrozdziałach, służą do transferu danych między poszczególnymi komponentami systemu. Od ich parametrów zależy sprawność komunikacji między peryferiami - źle skonfigurowana magistrala może spowalniać działanie całego systemu, przykładowo poprzez ograniczenie przepływności danych.

W celu wykorzystania danej szyny, wszystkie komponenty muszą mieć zaimplementowany interfejs, który pozwala na podłączenie się do danej magistrali. Jeśli któreś komponenty implementują interfejs innego mostka niż pozostałe, to można wykorzystać mostki, które pozwalają na konwersję między danymi standardami magistral.

\todo[inline]{Topologie?}

Istnieją różne standardy magistrali przeznaczone dla układów SoC, z czego bardziej znanymi są:

\begin{itemize}
	\item AMBA AXI od Arm Ltd.,
	\item Avalon autorstwa firmy Altera (obecnie Intel),
	\item Wishbone od Silicore Corporation (obecnie specyfikacja pod opieką Free and Open Source Silicon Foundation)
\end{itemize}

W ramach tej pracy wykorzystywana będzie magistrala Wishbone ze względu na w pełni dostępną publicznie, wolną od tantiemów specyfikację.

\todo[inline]{Coś więcej o szynie wishbone (szczegóły implementacyjne dwa rozdziały dalej)?}
\section{Projektowanie, testowanie i uruchamianie układu SoC}

\subsection{Konstruowanie układu SoC z użyciem frameworka LiteX}

O ile układ SoC można zaprojektować w całości od podstaw, częstą praktyką jest wykorzystywanie gotowych komponentów, często napisanych przez zewnętrznych dostawców, w celu szybszego zaprojektowania układu spełniającego założenia projektowe.

Framework LiteX\cite{https://doi.org/10.48550/arxiv.2005.02506} dostarcza różnego rodzaju komponenty oraz narzędzia pozwalające na zbudowanie zarówno kompletnego układu SoC jak i bardziej specjalizowanych systemów w formie syntezowalnej logiki. W tym celu wykorzystuje bibliotekę Migen, która implementuje język opisu sprzętu w formie obiektów w języku Python. Dzięki temu możliwe jest opisanie logiki wykorzystując popularny język programowania wykorzystujący znajomą składnię, co zmniejsza barierę wejścia; oprócz tego można wykorzystać narzędzia z ekosystemu języka Python, dzięki czemu dostępne są między innymi większe możliwości parametryzacji logiki czy systemy do przeprowadzania testów automatycznych. Jednocześnie możliwe jest łączenie logiki napisanej w innych językach, co ułatwia wykorzystywanie komponentów nie będących częścią frameworka.

\todo[inline]{RYSUNEK: Budowa frameworka LiteX}

LiteX został wybrany jako podstawa tego projektu ze względu na dużą popularność tego frameworka w otwarto-źródłowych projektach wykorzystujących układy FPGA oraz z powodu doświadczenia autora niniejszej pracy nabytej poprzez wykorzystanie frameworka w swoich poprzednich projektach.

\subsection{Symulowanie układu z użyciem LiteX i Verilator}

LiteX zawiera modularne środowisko symulacji, które pozwala na symulację zaprojektowanego układu na podstawie opisanej w języku Migen logiki oraz parametrów dotyczących środowiska zewnętrznego. Możliwe jest dzięki temu przetestowanie układu oraz przeznaczonego na nie oprogramowania, przykładowo poprzez wykorzystanie wirtualnego interfejsu Ethernet do komunikacji. Ułatwia to usuwanie błędów z projektu układu jak i oprogramowania, zanim jeszcze układ zostanie zsyntetyzowany do formy bitstreamu dla układu FPGA na potrzeby dalszych testów.

\todo[inline]{LISTING: targetowanie symulatora}

W celu uruchomienia naszego systemu w symulatorze, układ musi być skonfigurowany do zbudowania z użyciem narzędzi symulacyjnych zamiast narzędziami syntezy dedykowanych konkretnemu układowi FPGA. W tym celu LiteX integruje obsługę programu Verilator w celu konwersji układu do kodu komputerowego w formie źródłowej. Ów program linkowany jest z tak zwanym "wrapperem", który łączy naszą logikę z dodatkowymi modułami, które symulują wybrane peryferia z użyciem zasobów systemu operacyjnego, na którym symulacja będzie wykonywana. W przypadku tego projektu będzie to interfejs szeregowy UART dostępny w formie standardowego strumienia wejścia/wyjścia oraz niewykorzystany interfejs Ethernet w formie wirtualnego interfejsu sieciowego w systemie hosta. W ten sposób symulacja układu nie różni się znacząco od wykonywania programu komputerowego, co ułatwia interakcję z programem wykonywanym w wirtualnym środowisku.

\todo[inline]{LISTING: definiowanie I/O z użyciem dicta sim_config przekazywanego do buildera}

W praktyce zbudowanie symulatora układu polega na podmianie wybranych peryferiów na modele, które są częściowo zaimplementowane w formie bibliotek dla symulatora Verilator, oraz na opisaniu wykorzystywanych modułów i sposobu dostępu do nich z poziomu systemu operacyjnego hosta.

\subsection{Testowanie logiki z użyciem biblioteki cocotb}

Standardową praktyką przy testowaniu układów logicznych jest pisanie tak zwanych test benchy, czyli skryptów wykorzystywanych do testowania logiki poprzez sterowanie sygnałami wejściowymi oraz porównywanie stanu sygnałów wyjściowych z oczekiwanymi wzorcami. Owe skrypty są zwykle pisane w tym samym języku, co testowany układ logiczny.

Ze względu na wykorzystywanie języka Python w tej pracy do zaprojektowania układu SoC, dobrym pomysłem jest użycie go również w celu testowania logiki. W tym celu wykorzystana została biblioteka cocotb, która pozwala na testowanie projektów zaimplementowanych w językach (System)Verilog i VHDL z użyciem skryptów napisanych w języku Python. Cocotb współpracuje z zarówno komercyjnymi jak i otwarto-źródłowymi symulatorami, dzięki czemu można tą bibliotekę wykorzystać w istniejących już środowiskach do projektowania układów logicznych.

\todo[inline]{LISTING: przykładowy test}

Jako, że ręczne wykonywanie transmisji na magistrali Wishbone oraz interpretowanie ich wyników będzie często powtarzaną operacją, w zamian użyta została biblioteka cocotbext-wishbone, która udostępnia dwa obiekty: monitora oraz sterownika. Sterownik odpowiada stronie inicjującej transfer na magistrali, zaś monitor obsługuje transfery w sposób identyczny do peryferiów, w stronę których transfery są wykonywane.

\todo[inline]{LISTING: cocotbext-wishbone - przykład użycia}

Wspomniana biblioteka cocotbext-wishbone została rozbudowana o przekazywanie i monitorowanie stanu dodatkowych sygnałów, które zostały uprzednio zdefiniowane w specyfikacji, a nie były wykorzystywane przy standardowych transferach. Pozwoliło to na kontrolowanie i monitorowanie stanu magistrali w testowanych układach po dokonaniu w nich zmian opisanych w następnym rozdziale.

\todo[inline]{LISTING: cocotbext-wishbone - diff z dodania sygnałów?}

\subsection{Wnioski związane z doborem narzędzi}
\section{Realizacja oraz testowanie optymalizacji wybranej magistrali systemowej}
Ze względu na to, jak ważną rolę pełnią magistrale w systemie mikrokomputerowym, w tym rozdziale zostanie przeanalizowana specyfikacja magistrali Wishbone oraz jej implementacja w frameworku LiteX. Celem jest znalezienie części specyfikacji, które nie zostały zaimplementowane w LiteX, a które mogłyby przyspieszyć działanie systemu zbudowanego z jego wykorzystaniem.

\subsection{Budowa magistrali Wishbone}
Specyfikacja magistrali Wishbone definiuje wiele sygnałów, z których w praktyce wykorzystuje się wyłącznie część z nich. W podstawowych zastosowaniach wykorzystywane są następujące sygnały, opisywane z perspektywy inicjatora (mastera):
\begin{itemize}
    \item sygnały zegarowe i resetujące CLK oraz RST,
    \item szyna adresowa ADR_O,
    \item szyny danych DAT_I oraz DAT_O,
    \item sygnały synchronizacji CYC_O, STB_O oraz ACK_I,
    \item sygnał odczytu/zapisu WE_O,
    \item selektor bajtów słowa SEL_O
\end{itemize}

Szerokość szyn danych zależy od wielkości słowa, jakie mają być przesyłane w jednym cyklu zegara. Szerokość szyny adresowej zależy od tego, jaką przestrzeń adresową chcemy zaadresować. Magistrala wykorzystuje adresowanie po słowach - jeśli pożądana jest operacja na mniejszej jednostce, np. bajcie, wykorzystywany jest sygnał SEL_O w celu wybrania fragmentu słowa na szynie danych, które druga strona może wykorzystać.

\subsection{Rodzaje cykli na szynie Wishbone}

\subsubsection{Wishbone Classic Bus Cycles}
Podstawowy rodzaj cykli na magistrali Wishbone obsługuje operacje odczytu lub zapisu pojedynczego słowa szyny w trakcie pojedynczego cyklu transferu.

Cykl rozpoczyna się poprzez asercję sygnału $CYC$ przez inicjatora cyklu, a kończy poprzez jego negację. Sygnał $STB$ zaś informuje o gotowości do transferu słowa i jest utrzymywany w stanie wysokim, dopóki jeden z sygnałów terminujących cykl ($ACK$, $ERR$, $RTY$) nie zostanie zasertowany przez drugą stronę połączenia. Pozwala to na na regulację prędkości transferu przez obie strony połączenia.

\todo[inline]{DIAGRAM: przebieg odczytu pojedynczego słowa na magistrali}

W celu transferu słowa wykorzystywane są przynajmniej następujące sygnały: linia adresowa ($ADR$), linia danych ($DAT_I$, $DAT_O$), linia wyboru fragmentu słowa ($SEL$) oraz sygnał kierunku operacji ($WE$). Strona inicjująca połączenie na początku cyklu wystawia adres, na którym ma być wykonywana operacja; słowo na szynie $DAT_O$, jeśli jest to operacja zapisu; linie wybierające, które części szyny danych będą wykorzystywane w transferze oraz kierunek operacji. W następnym takcie zegara druga strona może wystawić słowo na szynie wejściowej danych ($DAT_I$), jesli jest to operacja zapisu, oraz zasercjować jeden z sygnałów synchronizujących - $ACK$, który potwierdza realizację operacji; $ERR$, informujący o wystąpieniu błędu w trakcie transferu; oraz $RTY$, który informuje o chwilowej braku gotowości strony odbiorczej do obsługi transferu.

\todo[inline]{DIAGRAM: przebieg zapisu pojedynczego słowa na magistrali}

W celu ograniczenia opóźnień przy transferze słów między kontrolerem a peryferium wykorzystuje się synchroniczne sygnały terminujące cykl. Wprowadza to jednak dodatkowe opóźnienie przy każdym słowie, gdyż peryferium nie jest świadome tego, że jest ponownie adresowane przy każdym kolejnym słowie - z tego powodu po odebraniu lub wystawieniu słowa peryferium kończy cykl poprzez negację sygnału $ACK$. Oznacza to, że transfer każdego słowa w cyklu przypadają dwa takty zegara, zakładając, że peryferium nie potrzebuje dodatkowych opóźnień z innych powodów.

\subsubsection{Wishbone Registered Feedback Bus Cycles}

Rozwiązaniem problemu cykli opóźniających przy transferze każdego kolejnego słowa jest rozbudowa magistrali o dodatkowe sygnały sterujące, które pozwoliłyby na informowanie peryferiów, że w danym cyklu magistrali nastąpi transfer większej ilości słów. Dzięki temu peryferium może się przygotować do odebrania większej ilości danych bez konieczności potwierdzania odbioru po każdym słowie.

\todo[inline]{DIAGRAM: uproszczony przebieg potokowego transferu danych}

Ten rodzaj sychronicznego terminowania cykli jest opisany w części specyfikacji "Wishbone Registered Feedback Bus Cycles". Dodaje ona dwa sygnały do magistrali, które są wykorzystywane do przekazywania informacji o rodzaju cyklu oraz związanego z nim dodatkowego parametru.

Sygnał CTI (Cycle Type Identifier) służy do identyfikowania rodzaju obecnie wykonywanego cyklu. Peryferium może wykorzystać tą informację, by przygotować się na kolejne części cyklu.
\begin{table}
\begin{center}
\begin{tabular}{ r|l }
 CTI[2:0] & opis \\ 
 \hline
 0b000 & cykl klasyczny \\
 0b001 & cykl potokowy ze stałym adresem \\
 0b010 & cykl potokowy z inkrementacją adresu \\
 0b011 & zarezerwowane \\
 0b100 & zarezerwowane \\
 0b101 & zarezerwowane \\
 0b110 & zarezerwowane \\
 0b111 & zakończenie cyklu \\
 \hline
\end{tabular}
\end{center}
\caption{\label{tab:signal-states-cti}Rodzaje stanów sygnału CTI}
\end{table}

Sygnał BTE (Burst Type Extension) zawiera zaś dodatkowe informacje zależne od danego rodzaju aktualnego cyklu.
W obecnej wersji specyfikacji jest wykorzystywany tylko w cyklu potokowym z inkrementacją adresu; jego wartość dla tego rodzaju cyklu można interpretować w następujący sposób:

\begin{table}
\begin{center}
\begin{tabular}{ r|l }
 BTE[1:0] & opis \\ 
 \hline
 0b00 & liniowa inkrementacja adresu \\
 0b01 & inkrementacja adresu z zawijaniem 2 najmniej znaczących bitów \\
 0b10 & inkrementacja adresu z zawijaniem 3 najmniej znaczących bitów \\
 0b11 & inkrementacja adresu z zawijaniem 4 najmniej znaczących bitów \\
 \hline
\end{tabular}
\end{center}
\caption{\label{tab:signal-states-bte}Rodzaje stanów sygnału BTE dla CTI równego 0b010}
\end{table}

Cykle potokowe z zawijającą inkrementacją adresu są zwykle wykorzystywane przy wypełnianiu linii pamięci podręcznej (cache) procesora; pozwala to na pełne pokrycie linii pamięci podręcznej, gdy adres, od którego zaczyna się dany cykl potokowy, nie pokrywa się z szerokością linii pamięci cache (która mieści kilka słów transferowych, przez co adresowana jest z pominięciem kilku najmniej znaczących bitów adresu)
Cykle potokowe ze stałym adresem przydatne są zaś przy strumieniowaniu danych do konkretnego adresu pojedynczego peryferium (na przykład przetwornika cyfrowo-analogowego) poprzez wykorzystanie dedykowanego podzespołu do szybkich transferów danych, zwanych również z angielskiego Direct Memory Access.

\subsection{Implementacja podzespołu z obsługą cykli transmisji seryjnej do stałego adresu}

W celu zaznajomienia się z projektowaniem peryferiów komunikujących się poprzez interfejs Wishbone, pierwszym projektem było stworzenie minimalnego peryferium, które oprócz interfejsu Wishbone ma również wyprowadzenia, które można wykorzystać w celu komunikacji z podzespołem drogą inną niż poprzez magistralę Wishbone. Pozwoliłoby to na weryfikację poprawności działania komunikacji poprzez porównywanie danych, które zostały zapisane lub odebrane z kontrolera poprzez magistralę z tymi, które można odebrać lub nadać do peryferium inną drogą.

\todo[inline]{DIAGRAM: schemat blokowy fifo_transceiver.py}

Peryferium składa się z dwóch synchronicznych kolejek FIFO oraz logiki obsługującej magistralę Wishbone. Jedna kolejka FIFO służy do odbioru danych z zewnętrznego źródła, druga zaś odbiera dane z zewnętrznego źródła. Obie kolejki zostały podłączone do magistrali Wishbone z użyciem logiki zawierającej maszynę stanów skończonych, która zajmuje się sterowaniem sygnałami synchronizacji portów wewnętrznych kolejek w zależności od rodzaju operacji na magistrali (odczyt lub zapis). Zewnętrzne porty kolejek zostały wyprowadzone na zewnątrz peryferium bez dodatkowej logiki sterującej w celu ułatwienia przeprowadzenia testów na dalszych etapach pracy. Kolejki są dostępne jako gotowy element w module $migen.genlib$ będącej częścią biblioteki do opisu logiki Migen.

\todo[inline]{DIAGRAM: schemat maszyny stanowej bez burstów}

Maszyna stanowa składa się z trzech stanów: $IDLE$, $READ$ oraz $WRITE$.
Stanem startowym jest $IDLE$. Gdy zostanie zainicjowany cykl transferu danych na magistrali poprzez zasertowanie sygnałów CYC oraz STB, maszyna w kolejnym takcie zegara przejdzie do stanu $READ$ lub $WRITE$, w zależności od stanu sygnału WE na magistrali.
Przy stanie $WRITE$ sprawdzany jest stan sygnału $writable$ na wewnętrznym porcie wyjściowej kolejki FIFO. Jeśli kolejka nie jest zapełniona, są asertowane sygnały: $WE$ portu zapisu kolejki, powodujący zapisanie słowa wystawionego na magistrali do kolejki oraz $ACK$ magistrali, który służy do informowania kontrolera szyny, że transmisja została potwierdzona. W przypadku, gdy kolejka jest zapełniona, sygnały $ACK$ magistrali oraz $WE$ kolejki nadawczej są negowane - w ten sposób kontroler magistrali wie, że peryferium nie może potwierdzić przetworzenia odebranego słowa.
Podobnie funkcjonuje to przy stanie $READ$ maszyny, jednak w tym wypadku potwierdzenie transmisji zależy od tego, czy w kolejce znajduje się przynajmniej jedno słowo, które można odczytać z kolejki - o obecności danych w kolejce informuje sygnał $readable$ kolejki odbiorczej, zaś potwierdzenie odczytania słowa w kolejce jest potwierdzane poprzez zasertowanie sygnału $RE$ kolejki odbiorczej.

Z reguły 4.35 opisanej w specyfikacji interconnectu Wishbone wynika fakt, że w trakcie trwania cyklu potokowego ze stałym adresem nie mogą zmienić się następujące sygnały magistrali: linia adresowa, linia wybierająca $SEL$ oraz wybór operacji $WE$. Wobec tego można założyć, że cykl rozpoczęty operacją zapisu będzie w kolejnych taktach zegara przekazywał kolejne słowa do odebrania przez podzespół.

\todo[inline]{DIAGRAM: schemat maszyny stanowej z podkreślonymi akcjami przy CTI == 0b001}

W celu kontynuowania cyklu przy każdym takcie zegara sprawdzane są następujące warunki: czy cykl dalej jest aktywny (sygnały $CYC$ i $STB$ są aktywne) oraz czy sygnał przekazujący rodzaj cyklu $CTI$ ma dalej wartość $0b001$, oznaczający cykl potokowy ze stałym adresem. Jeśli oba te warunki są spełnione, to ponawiamy obecny stan maszyny stanów w następnym takcie zegara, podtrzymując sygnał ACK w stanie wysokim. Dzięki temu w każdym takcie zegara kolejne słowa zostaną odebrane z kolejki odbiorczej albo dodane do kolejki nadawczej w podzespole, bez inicjalizowania cyklu magistrali na nowo.

\todo[inline]{DIAGRAM: przebiegi prezentujące różnicę (albo to dopiero przy testowaniu pokazać)}

\subsection{Rozbudowanie istniejącego podzespołu (SRAM) o obsługę cykli transmisji seryjnej z inkrementacją adresu}

Peryferium SRAM, będący częścią biblioteki rdzeni w frameworku LiteX, składa się z dwóch części: synchronicznej pamięci (z biblioteki Migen) o zdefiniowanej szerokości słowa i pojemności (liczonej w słowach) oraz z interfejsu Wishbone, który łączy wspomnianą pamięć z magistralą.

\todo[inline]{DIAGRAM: soc/interconnect/wishbone.WishboneSRAM - schemat budowy}

Interfejs implementuje wyłącznie podstawowy rodzaj cykli - jako, że wybrany procesor obsługuje potokowe transfery danych, będziemy w stanie, po rozbudowaniu implementacji interfejsu w module SRAM, zweryfikować wpływ obsługi transferów potokowych na praktyczną wydajność systemu mikroprocesorowego.

W celu dodania obsługi transferów potokowych do modułu SRAM należy rozbudować interfejs o następujące elementy: generator adresów, wykorzystywany do wyznaczania kolejnych adresów pamięci w każdym cyklu zegara; dodatkowy warunek w logice kombinatoryjnej odpowiedzialnej za generowanie sygnału potwierdzenia ACK; logika przekazywania wyznaczonych adresów.

Zakładamy, że wewnętrzny moduł pamięci RAM jest w stanie wykonywać kolejne operacje w każdym takcie zegara, tj. w każdym takcie zegara można wykonać operację odczytu lub zapisu słowa pod dowolnym adresem.
Kolejnym założeniem, wynikającym ze specyfikacji magistrali, jest niezmienność sygnałów CTI oraz BTE informujących o rodzaju cyklu w czasie jego trwania, z wyłączeniem stanu sygnału CTI informującym o zakończeniu cyklu potokowego.

\subsubsection{Detekcja rodzaju cyklu}

Strona inicjująca połączenie informuje o rodzaju cyklu poprzez sygnał CTI. Jeśli jedna ze stron transferu nie obsługuje wybranego typu cyklu, wtedy transfer przebiega jak w cyklu klasycznym (jeden transfer - jeden cykl).
Z dwóch zdefiniowanych w specyfikacji rodzajów cykli potokowych, ze stałym adresem oraz inkrementacją adresu, w przypadku pamięci RAM interesuje nas cykl potokowy z inkrementacją adresu. W tym celu wykorzystamy warunek kombinacyjny, który w wypadku wystąpienia wspieranego rodzaju cyklu asertuje sygnał informujący o żądaniu transferu potokowego.
Oprócz tego definiujemy wektor adr_counter_base, w którym będzie zawarty adres bazowy z wyłączeniem bitów, które będą wykorzystywane w przypadku inkrementacji adresu z zapętleniem najmniej znaczących bitów.

\begin{listing}[H]
\begin{minted}{python}
class SRAM(Module):
    def __init__(self, mem_or_size, read_only=None, init=None, bus=None, burst=True):
            # ...
            adr_burst = Signal()

            # Obsługiwane są tylko cykle z inkrementacją adresu
            self.comb += [
                Case(self.bus.cti, {
                    # cykl z inkrementacją adresu
                    0b010: adr_burst.eq(1),
                    # zakończenie cyklu
                    0b111: adr_burst.eq(0),
                    # pozostałe rodzaje cykli (w tym cykl standardowy)
                    "default": adr_burst.eq(0)
                }),
            ]
\end{minted}
\caption{\label{lst:impl-sram-detect}Logika kombinacyjna generująca sygnał informujący o cyklu potokowym na podstawie sygnału CTI}
\end{listing}

\subsubsection{Generator sekwencji adresów}

Gdy cykl zostanie rozpoczęty poprzez zasertowanie sygnałów CYC i STB oraz inicjator połączenia zażąda transferu potokowego, inicjalizowany jest licznik adresów. Jest on wykorzystywany do generowania kolejnych adresów, do których dostęp strona inicjalizująca połączenie chce mieć dostęp w kolejnych taktach zegara bez oczekiwania na podanie kolejnych adresów. Gdy adres został zapisany w liczniku, w kolejnych taktach zegara licznik będzie inkrementowany o 1.
Jednocześnie w dodatkowym rejestrze, nazwanym $adr_counter_offset$ zapisywane są dolne bity pierwszego zażądanego adresu. Zawartość tego dodatkowego rejestru będzie wykorzystywana do wyliczenia dolnych bitów dolnego adresu, jeśli inkrementacja adresu ma następywać z zawijaniem dolnych bitów.

\begin{longlisting}
\begin{minted}{python}
class SRAM(Module):
    def __init__(self, mem_or_size, read_only=None, init=None, bus=None, burst=True):
        # ...
        adr_wrap_mask = Array((0b0000, 0b0011, 0b0111, 0b1111))
        adr_wrap_max = adr_wrap_mask[-1].bit_length()

        adr_latched = Signal()

        adr_counter = Signal(len(self.bus.adr))
        adr_counter_offset = Signal(adr_wrap_max)

        # Zapisanie adresu startowego nowego cyklu
        self.sync += [
            If(self.bus.cyc & self.bus.stb & adr_burst,
                adr_latched.eq(1),
                # Jeśli adres został zapisany, inkrementuj go w każdym takcie zegara
                If(adr_latched,
                    adr_counter.eq(adr_counter + 1)
                ).Else(
                    # Zapisz najmniej znaczące bity adresu w wypadku cyklu
                    # z zawijaniem adresu
                    adr_counter_offset.eq(self.bus.adr & adr_wrap_mask[self.bus.bte]),
                    # Zapisz adres bez najmniej znaczących bitów,
                    # jeśli trwa cykl z zawijaniem adresu. Jeśli będzie to operacja odczytu,
                    # licznik będzie preinkrementowany o 1
                    If(self.bus.we,
                        adr_counter.eq(
                            Cat(self.bus.adr & ~adr_wrap_mask[self.bus.bte],
                                self.bus.adr[adr_wrap_max:]
                                )
                        ),
                    ).Else(
                        adr_counter.eq(
                            Cat(self.bus.adr & ~adr_wrap_mask[self.bus.bte],
                                self.bus.adr[adr_wrap_max:]
                                ) + 1
                        ),
                    )
                ),
                If(self.bus.cti == 0b111,
                    adr_latched.eq(0),
                    adr_counter.eq(0),
                    adr_counter_offset.eq(0)
                )
            # Zerowanie licznika po zakończeniu cyklu
            ).Else(
                adr_latched.eq(0),
                adr_counter.eq(0),
                adr_counter_offset.eq(0)
            ),
        ]
\end{minted}
\caption{\label{lst:impl-sram-adrcounter}Fragment logiki synchronicznej odpowiedzialnej za kontrolę licznika adresu}
\end{longlisting}

Gdy inicjator cyklu informuje o zakończeniu transferu potokowego poprzez ustawienie odpowiedniej wartości na sygnale CTI, licznik jest zerowany i zatrzymywany, by przy kolejnych taktach zegara nie wykonywać nadmiarowych operacji w pamięci.

W części kombinacyjnej generatora następny adres jest wyznaczany poprzez zsumowanie dwóch sygnałów, gdzie jeden sygnał zawiera wartość licznika bez zawijanych bitów adresu, zaś drugi sygnał zawiera dolne bity adresu wyznaczone poprzez zsumowanie zawartości licznika i rejestru zawierającego dolne bity adresu początkowego, z których wykorzystywane są wyłącznie zawinięte bity.przebieg zapisu pojedynczego słowa na magistrali

\begin{listing}[H]
\begin{minted}{python}
class SRAM(Module):
    def __init__(self, mem_or_size, read_only=None, init=None, bus=None, burst=True):
        # ...
        adr_offset_lsb = Signal(adr_wrap_max)
        adr_offset_msb = Signal(len(self.bus.adr))
        adr_next = Signal(len(self.bus.adr))

        # Wyznacz kolejny adres poprzez zsumowanie zawartości licznika
        # z przesunięciem wyznaczonym przez najmniej znaczące bity adresu startowego
        self.comb += [
            adr_offset_lsb.eq((adr_counter + adr_counter_offset) & adr_wrap_mask[self.bus.bte]),
            adr_offset_msb.eq(adr_counter & ~adr_wrap_mask[self.bus.bte]),
            adr_next.eq(adr_offset_msb + adr_offset_lsb)
        ]
\end{minted}
\caption{\label{lst:impl-sram-adrnext}Logika kombinacyjna odpowiedzialna za sumowanie docelowego adresu w każdym cyklu zegara}
\end{listing}

Wyznaczany adres, jeśli licznik jest uruchomiony, jest przekazywany w logice kombinacyjnej do linii adresowej modułu pamięci RAM - w przeciwnym wypadku adres do pamięci RAM jest przekazywany bezpośrednio z magistrali, wliczając pierwszy adres przy transferach potokowych.

\subsubsection{Generowanie sygnału potwierdzenia}

Logika synchroniczna odpowiedzialna za generowanie sygnału ACK została zmodyfikowana w nieznaczny sposób, poprzez dodanie warunku OR dla znegowanego sygnału ACK z poprzedniego taktu oraz wewnętrznego sygnału informującego o aktywnym transferze potokowym. W ten sposób sygnał ACK jest aktywny w każdym takcie zegara przy transferach potokowych, zaś przy klasycznych cyklach zachowanie jest to samo co poprzednio, czyli sygnał ACK jest natychmiastowo negowany w drugim takcie zegara po rozpoczęciu transferu.

\begin{listing}[H]
\begin{minted}{python}
class SRAM(Module):
    def __init__(self, mem_or_size, read_only=None, init=None, bus=None, burst=True):
        # ...
        self.sync += [
            self.bus.ack.eq(0),
            If(self.bus.cyc & self.bus.stb & (~self.bus.ack | adr_burst),
                self.bus.ack.eq(1)
            )
        ]
\end{minted}
\caption{\label{lst:impl-sram-adrnext}Fragment logiki synchronicznej generującej sygnał ACK}
\end{listing}

W celu uproszczenia kodu sygnał $adr_burst$ jest zawsze generowany w układzie, nawet jesli obsługa transferów potokowych zostanie wyłączana - w tym wypadku sygnał ten ma zawsze wartość $0$; sygnał ten może zostać automatycznie usunięty w trakcie optymalizacji układu przez programy do syntezy logiki.

\subsection{Wnioski po optymalizacji wybranego peryferium}

Dzięki prostocie implementacji interfejsu magistrali w module pamięci SRAM jak i samego modułu wprowadzenie obsługi transferów potokowych nie wymagało większych zmian w logice interfejsu - całość została sprowadzona do zaimplementowania licznika adresu po stronie modułu oraz dodatkowych warunków w logice synchronizującej komunikację po magistrali Wishbone.
\todo[inline]{Co jeszcze można dodać?}

\section{Realizacja testów i analiza ich wyników}
TODO: co będzie robione w celu przeanalizowania wpływu zmian w logice peryferiów?

\subsection{Analiza i porównanie przebiegów aktywności magistrali}
TODO: po co w ogóle te wszystkie testy?

\subsubsection{Implementacja testów jednostkowych dla peryferiów}
TODO: Opisanie procesu testowania peryferiów (harness i same testy)

\subsubsection{Porównanie wyników testów}
TODO: przebiegi dla komponentów z c3 przed i po dodaniu burstów

\subsection{Testy wydajnościowe systemu jednoukładowego na symulatorze oraz na układzie FPGA}
TODO: wstęp - opisanie procesu benchmarkowania, platformy docelowe

\subsubsection{Program na platformę RISC-V badający szybkość operacji na pamięci}
TODO: opisz co kod robi - memtest i memspeed, trochę o tracingu

\subsubsection{Wyniki testów wydajnościowych}
TODO: tabelki i wstępne przemyślenia
\section{Podsumowanie}

Na początku pracy omówione zostały systemy jednoukładowe, zaczynając od historii komputerów jednoukładowych oraz wymienienia popularnych obecnie urządzeń bazujących na układach System on Chip. Po określeniu w ten sposób przykładów takich systemów możliwym było określenie wspólnych cech, które definiują czym dokładnie jest system jednoukładowy. Określona została również wybrana implementacja procesora zgodnego z architekturą RISC-V oraz standardu magistrali systemowej, na których została oparta platforma testowa stworzona na potrzeby niniejszej pracy.

Opisane zostało zoptymalizowanie wydajności systemu jednoukładowego poprzez zaimplementowanie obsługi transferów potokowych w module pamięci SRAM. Optymalizacja ta pozwoliła na przyspieszenie odczytu danych z pamięci SRAM o przynajmniej 23 procent bez modyfikacji innych parametrów pracy systemu, takich jak częstotliwość taktowania. Było to możliwe dzięki przeanalizowaniu możliwości wybranego procesora oraz zaznajomieniu się z różnymi trybami pracy opisanymi w specyfikacji magistrali systemowej; jednocześnie możliwym jest dalsze analizowanie możliwości optymalizacji systemów jednoukładowych w celu osiągnięcia lepszych parametrów wydajnościowych.

Zarówno moduł SRAM, będący częścią większego środowiska LiteX wykorzystywanego do tworzenia systemów jednoukładowych, jak i testy jednostkowe wykorzystane w celu sprawdzenia poprawności działania zmodyfikowanego układu, zostały zaimplementowane w tym samym języku programowania o nazwie Python. Umożliwiło to wykorzystanie tych samych narzędzi, takich jak systemy testów jednostkowych, których używa się do rozwoju oprogramowania w tym języku, zmniejszając barierę wejścia w porównaniu z narzędziami dostępnymi dla częściej wykorzystywanych języków opisu sprzętu, takich jak Verilog czy VHDL. Jednocześnie, dzięki integracji środowiska LiteX z zamkniętymi narzędziami dedykowanymi dla konkretnych układów reprogramowalnych, możliwym było uruchomienie platformy testowej na fizycznym układzie FPGA.

Praca ta udowadnia, iż jest możliwym realizacja kompletnego systemu jednoukładowego z wykorzystaniem w większości otwartoźródłowych komponentów, co umożliwia ograniczenie kosztów związanych z projektowaniem i integracją takiego systemu. Opisana optymalizacja została włączona do publicznego kodu środowiska LiteX, co pozwala na wykorzystanie jej w rozwiązaniach zarówno hobbystycznych jak i komercyjnych.


\bibliographystyle{plain}
\bibliography{refs}

\end{document}
