\section{Cele i założenia pracy}

Celem niniejszej pracy jest rozbudowa implementacji wybranej magistrali systemowej o wsparcie dla potokowego transferu danych oraz analiza wpływu tej zmiany na wydajność całego systemu. Jest to realizowane w celu przyspieszenia operacji transferu danych między kontrolerem magistrali a peryferiami.

W pierwszej kolejności opisana zostanie architektura przykładowego układu typu System on Chip, z naciskiem na rdzeń procesora przeznaczenia ogólnego oraz na magistralę systemową.

Rozdział drugi będzie opisywać wybrane narzędzia wykorzystane do zrealizowania prostego układu SoC. Układ ten będzie wykorzystywany w formie symulatora uruchamianego na komputerze klasy PC, jak i logiki wykonywanej na układzie programowalnych macierzy bramek FPGA. Oprócz tego wymienione zostaną biblioteki wykorzystane przy realizacji testów jednostkowych używanych do weryfikacji poprawności działania wybranych peryferiów.

W rozdziale trzecim będzie przedstawiony opis rozbudowy implementacji magistrali systemowej Wishbone B4, zaczynając od doboru optymalizacji z dostępnej specyfikacji, kończąc zaś na praktycznych przykładach w rozbudowywanych peryferiach komunikujących się poprzez magistralę.

Przedostatni rozdział pracy zawierać będzie wyniki testów jednostkowych oraz wydajnościowych zrealizowanych z użyciem symulatora i układu FPGA, natomiast w ostatnim rozdziale opisane będą wnioski wyciągnięte z realizacji oraz testowania optymalizacji.
