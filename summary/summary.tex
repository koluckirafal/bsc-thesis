\section{Podsumowanie}

Na początku pracy omówione zostały systemy jednoukładowe, zaczynając od historii komputerów jednoukładowych oraz wymienienia popularnych obecnie urządzeń bazujących na układach System on Chip. Po określeniu w ten sposób przykładów takich systemów możliwe było określenie wspólnych cech, które definiują, czym dokładnie jest system jednoukładowy. Określona została również wybrana implementacja procesora zgodnego z architekturą RISC-V oraz standardu magistrali systemowej, na których została oparta platforma testowa stworzona na potrzeby niniejszej pracy.

Opisane zostało zoptymalizowanie wydajności systemu jednoukładowego poprzez zaimplementowanie obsługi transferów potokowych w module pamięci RAM. Optymalizacja ta pozwoliła na przyspieszenie odczytu danych z pamięci RAM o przynajmniej 23 procent bez modyfikacji innych parametrów pracy systemu, takich jak częstotliwość taktowania. Jednocześnie według testów jednostkowych słowa były przekazywane w niemal połowie mniejszej ilości cykli niż w wypadku pojedynczych transferów. Było to możliwe dzięki przeanalizowaniu możliwości wybranego procesora oraz zaznajomieniu się z różnymi trybami pracy opisanymi w specyfikacji magistrali systemowej; jednocześnie możliwym jest dalsze analizowanie możliwości optymalizacji systemów jednoukładowych w celu osiągnięcia lepszych parametrów wydajnościowych.

Zarówno moduł SRAM, będący częścią większego środowiska LiteX wykorzystywanego do tworzenia systemów jednoukładowych, jak i testy jednostkowe wykorzystane w celu sprawdzenia poprawności działania zmodyfikowanego układu, zostały zaimplementowane w tym samym języku programowania o nazwie Python. Umożliwiło to wykorzystanie tych samych narzędzi, takich jak systemy testów jednostkowych, których używa się do rozwoju oprogramowania w tym języku, zmniejszając barierę wejścia w porównaniu z narzędziami dostępnymi dla częściej wykorzystywanych języków opisu sprzętu, takich jak Verilog czy VHDL. Jednocześnie, dzięki integracji środowiska LiteX z zamkniętymi narzędziami dedykowanymi dla konkretnych układów FPGA, możliwym było uruchomienie platformy testowej na fizycznym układzie FPGA.

Praca ta udowadnia, iż jest możliwym realizacja kompletnego systemu jednoukładowego z wykorzystaniem w większości otwartoźródłowych komponentów, co umożliwia ograniczenie kosztów związanych z projektowaniem i integracją takiego systemu. Opisana optymalizacja została włączona do publicznego kodu środowiska LiteX, co pozwala na wykorzystanie jej w rozwiązaniach zarówno hobbystycznych, jak i komercyjnych.
