\section{Podsumowanie}

W niniejszej pracy opisane zostało zoptymalizowanie wydajności systemu jednoukładowego poprzez zaimplementowanie obsługi transferów potokowych w module pamięci SRAM. Optymalizacja ta pozwoliła na przyspieszenie odczytu danych z pamięci SRAM o przynajmniej 23 procent bez modyfikacji innych parametrów pracy systemu, takich jak częstotliwość taktowania. Założenia opisane w pierwszych rozdziałach tej pracy zostały osiągnięte; jednocześnie możliwym jest dalsze analizowanie możliwości optymalizacji systemów jednoukładowych w celu osiągnięcia lepszych parametrów wydajnościowych.

Zarówno moduł SRAM, będący częścią większego środowiska LiteX wykorzystywanego do tworzenia systemów jednoukładowych, jak i testy jednostkowe wykorzystane w celu sprawdzenia poprawności działania zmodyfikowanego układu, zostały zaimplementowane w tym samym języku programowania o nazwie Python. Umożliwiło to wykorzystanie tych samych narzędzi, których używa się do rozwoju oprogramowania w tym języku, zmniejszając barierę wejścia w porównaniu z narzędziami dostępnymi dla częściej wykorzystywanych języków opisu sprzętu, takich jak Verilog czy VHDL. Jednocześnie, dzięki integracji środowiska LiteX z zamkniętymi narzędziami dedykowanymi dla konkretnych układów reprogramowalnych, umożliwiając uruchomienie platformy testowej na fizycznym układzie FPGA.

Praca ta udowadnia, iż jest możliwym realizacja kompletnego systemu jednoukładowego z wykorzystaniem w większości otwartoźródłowych komponentów, co umożliwia ograniczenie kosztów związanych z projektowaniem i integracją takiego systemu. Opisana optymalizacja została włączona do publicznego kodu środowiska LiteX, co pozwala na wykorzystanie jej w rozwiązaniach zarówno hobbystycznych jak i komercyjnych.
